% Options for packages loaded elsewhere
\PassOptionsToPackage{unicode}{hyperref}
\PassOptionsToPackage{hyphens}{url}
%
\documentclass[
]{article}
\usepackage{amsmath,amssymb}
\usepackage{iftex}
\ifPDFTeX
  \usepackage[T1]{fontenc}
  \usepackage[utf8]{inputenc}
  \usepackage{textcomp} % provide euro and other symbols
\else % if luatex or xetex
  \usepackage{unicode-math} % this also loads fontspec
  \defaultfontfeatures{Scale=MatchLowercase}
  \defaultfontfeatures[\rmfamily]{Ligatures=TeX,Scale=1}
\fi
\usepackage{lmodern}
\ifPDFTeX\else
  % xetex/luatex font selection
\fi
% Use upquote if available, for straight quotes in verbatim environments
\IfFileExists{upquote.sty}{\usepackage{upquote}}{}
\IfFileExists{microtype.sty}{% use microtype if available
  \usepackage[]{microtype}
  \UseMicrotypeSet[protrusion]{basicmath} % disable protrusion for tt fonts
}{}
\makeatletter
\@ifundefined{KOMAClassName}{% if non-KOMA class
  \IfFileExists{parskip.sty}{%
    \usepackage{parskip}
  }{% else
    \setlength{\parindent}{0pt}
    \setlength{\parskip}{6pt plus 2pt minus 1pt}}
}{% if KOMA class
  \KOMAoptions{parskip=half}}
\makeatother
\usepackage{xcolor}
\usepackage[margin=1in]{geometry}
\usepackage{graphicx}
\makeatletter
\def\maxwidth{\ifdim\Gin@nat@width>\linewidth\linewidth\else\Gin@nat@width\fi}
\def\maxheight{\ifdim\Gin@nat@height>\textheight\textheight\else\Gin@nat@height\fi}
\makeatother
% Scale images if necessary, so that they will not overflow the page
% margins by default, and it is still possible to overwrite the defaults
% using explicit options in \includegraphics[width, height, ...]{}
\setkeys{Gin}{width=\maxwidth,height=\maxheight,keepaspectratio}
% Set default figure placement to htbp
\makeatletter
\def\fps@figure{htbp}
\makeatother
\setlength{\emergencystretch}{3em} % prevent overfull lines
\providecommand{\tightlist}{%
  \setlength{\itemsep}{0pt}\setlength{\parskip}{0pt}}
\setcounter{secnumdepth}{5}
\newlength{\cslhangindent}
\setlength{\cslhangindent}{1.5em}
\newlength{\csllabelwidth}
\setlength{\csllabelwidth}{3em}
\newlength{\cslentryspacingunit} % times entry-spacing
\setlength{\cslentryspacingunit}{\parskip}
\newenvironment{CSLReferences}[2] % #1 hanging-ident, #2 entry spacing
 {% don't indent paragraphs
  \setlength{\parindent}{0pt}
  % turn on hanging indent if param 1 is 1
  \ifodd #1
  \let\oldpar\par
  \def\par{\hangindent=\cslhangindent\oldpar}
  \fi
  % set entry spacing
  \setlength{\parskip}{#2\cslentryspacingunit}
 }%
 {}
\usepackage{calc}
\newcommand{\CSLBlock}[1]{#1\hfill\break}
\newcommand{\CSLLeftMargin}[1]{\parbox[t]{\csllabelwidth}{#1}}
\newcommand{\CSLRightInline}[1]{\parbox[t]{\linewidth - \csllabelwidth}{#1}\break}
\newcommand{\CSLIndent}[1]{\hspace{\cslhangindent}#1}
\usepackage[utf8]{inputenc}
\usepackage{textcomp}
\usepackage{lmodern}
\ifLuaTeX
  \usepackage{selnolig}  % disable illegal ligatures
\fi
\IfFileExists{bookmark.sty}{\usepackage{bookmark}}{\usepackage{hyperref}}
\IfFileExists{xurl.sty}{\usepackage{xurl}}{} % add URL line breaks if available
\urlstyle{same}
\hypersetup{
  pdftitle={Investigating Microbial Community in an Anaerobic Reactor for the Treatment of Wastewater from a Molasses-Based Spirit and Yeast Production Factory},
  hidelinks,
  pdfcreator={LaTeX via pandoc}}

\title{Investigating Microbial Community in an Anaerobic Reactor for the
Treatment of Wastewater from a Molasses-Based Spirit and Yeast
Production Factory}
\author{Yamila Timmer\textsuperscript{1}, Floris
Menninga\textsuperscript{1}, Jarno Jacob Duiker\textsuperscript{1}\\
\(^1\)Hanze University of Applied Sciences, Life Sciences,
Bioinformatics, Groningen}
\date{March 30, 2025}

\begin{document}
\maketitle

{
\setcounter{tocdepth}{2}
\tableofcontents
}
\hypertarget{introduction}{%
\section{Introduction}\label{introduction}}

\textbf{Abstract}

This study employs metagenomic tools to analyze wastewater from the Afac
filtering lagoon in Kenya, which is used by an agrochemical company to
treat river water for production processes before release back into the
river. The goal is to evaluate the lagoon's effectiveness in removing
hazardous microorganisms and assess potential ecological impacts on
downstream ecosystems and communities.

Sequencing data from the MinION platform underwent rigorous quality
assessment and trimming using Fastplong (Chen et al. 2018), followed by
taxonomic classification with Kraken 2 (Wood, Lu, and Langmead 2019).
Results were visualized interactively using KronaTools (Ondov, Bergman,
and Phillippy 2011) and Pavian (Breitwieser and Salzberg 2020), enabling
detailed taxonomic profiling. Microbial functional pathways were
analyzed with HUMAnN 3.0 (Beghini et al. 2021), while alpha- and
beta-diversity metrics were calculated using a krona-plugin python
script.

Our analysis identifies microorganisms persisting through the lagoon's
treatment stages, providing insights into filtration efficacy and risks
of downstream pollution. The findings will serve as actionable
recommendations to optimize the lagoon system, mitigate ecological harm,
and safeguard river health for adjacent communities.

\textbf{Introduction}

Countries in northern Africa, the Middle East, Singapore, Maldives, and
Australia face water shortages. Making water quite a valuable resource,
because of this these countries and regions have practiced reclaiming
waste water from factories and sewage. Instead of wasting this water it
could help against the big shortages and help the infrastructure stay up
and active.

Reclaimed water is used in various places, most known is the landscape
irrigation to maintain green living spaces and for agricultural
irrigation to produce food (\textbf{Hong2020?}). Reclaimed water is also
key for important infrastructures like cooling towers that serve
electrical power plants. All these things mean that reclaiming this
waste water is key to some countries' daily operation.

Reclaiming waste water does come with some risks. The water used in the
factories may contain chemicals or microorganisms that could have a
negative influence on the health of people or the ecosystem
(\textbf{Chen2013?}). Many of these factories use filters to take these
chemicals and/or microorganisms out of the water so it can return safely
into the river or other water source, however not every factory uses the
same filter due to costs and/or the infrastructure to make these filters
is not available. Thorough filtering is key in reclaiming waste water,
because many biological contaminants are disseminated through water, and
their occurrence has potential detrimental impacts on public and
environmental health (\textbf{Hong2020?}).

\hypertarget{materials-and-methods}{%
\section{Materials and Methods}\label{materials-and-methods}}

\hypertarget{overall-approach}{%
\subsubsection{Overall Approach}\label{overall-approach}}

The sequencing data obtained from the MinION platform underwent a
quality assessment process. Initially, the sequences were evaluated for
quality using \texttt{Fastplong} (Chen et al. 2018). After this,
\texttt{Fastplong} also trimmed the data according to the generated
reports. Following this, a secondary quality assessment was conducted
using \texttt{Fastplong} to ensure improved data quality. The refined
dataset was then subjected to taxonomic classification through
\texttt{Kraken2} (Wood, Lu, and Langmead 2019), a computational tool for
microbial classification. The results generated by \texttt{Kraken2} were
then visualized using \texttt{KronaTools} (Ondov, Bergman, and Phillippy
2011) and \texttt{Pavian} (Breitwieser and Salzberg 2020), facilitating
an interactive and intuitive representation of the taxonomic
distribution. Other tools like \texttt{HUMAnN\ 3.0} (Beghini et al.
2021) were used for profiling the abundance of microbial metabolic
pathways and other molecular functions from metagenomics data.
Furthermore, the outcomes obtained from the \texttt{Kraken2} analysis
were employed to identify the microorganisms present in the lagoon
water, which allowed comparing the different lagoon stages and their
microbiomes.

To get more insight into the data, we used two Kraken2 plugins that
calculate the alpha- and beta-diversity, including a tool that
calculated the Shannon index for alpha diversity:

\textbf{Shannon index for alpha diversity}

\[
H = -\sum_{i=1}^{S} p_i \ln p_i
\]

where: - \(H\) is the Shannon diversity index - \(S\) is the total
number of species - \(p_i\) is the proportion of individuals that belong
to species \(i\)

This index showed the species diversity in each sample and the
distribution of the species in the samples. With all this information,
an analysis for each stage of the lagoon has been made.

\hypertarget{data-collection}{%
\subsubsection{Data Collection}\label{data-collection}}

\hypertarget{qiaamp-dna-microbiome-kit-50---51704}{%
\paragraph{QIAamp DNA Microbiome Kit (50) -
51704}\label{qiaamp-dna-microbiome-kit-50---51704}}

This kit is used for purification and enrichment of bacterial microbiome
DNA from swabs (and body fluids). Effective depletion of host DNA during
the purification process maximizes bacterial DNA coverage in NGS
analysis and allows for 16rDNA-based microbiome analysis and whole
metagenome shotgun sequencing studies.

\textbf{Procedure}

The kit employs spin column technology with a specialized protocol to
enrich bacterial microbiome DNA while minimizing host DNA contamination.
First, host cells are gently lysed, and their released DNA is
enzymatically degraded. Next, bacterial cells are disrupted using
optimized mechanical and chemical lysis. The bacterial DNA is then
selectively bound to a silica membrane, purified through washes, and
finally eluted for analysis.

This method ensures efficient isolation of bacterial DNA from complex
samples, reducing host DNA interference.

\hypertarget{sequencing}{%
\paragraph{Sequencing}\label{sequencing}}

We used Rapid sequencing amplicons - 16S barcoding (SQK-16S024).

Note: The flow cell used in our MinION was not of the highest
quality/able to produce a good read amount for our last sample. This
resulted in only the Lagoon in and out samples being usable.

The MinION Flow Cell can generate up to 50 Gb of data for sequencing
DNA, cDNA or native RNA in real-time. A flowcell is a core sensing unit
made up of nanopores, an array of microscaffolds that supports membrane
and embedded nanopore. The array keeps the multiple nanopores stable
during shipping and usage. Each microscaffold corresponds to its own
electrode that is connected to a channel in the sensor array chip.
Sensor arrays may be manufactured with any number of channels and an
ASIC (Application-Specific Integrated Circuit), with each nanopore
channel being controlled and measured individually by the bespoke ASIC.
This allows for multiple nanopore experiments to be performed in
parallel.

More information is available at:
\url{https://nanoporetech.com/platform/technology/flow-cells-and-nanopores}

Using MinKNOW, the sequencing was started. To configure the settings
needed for MinION sequencing, the user had to go to the kit page in
MinKNOW and select the kit used for the library preparation.

The risk of using an old MinION flowcell is that the flowcell can change
over time, and the change between each sequencing run is hard to
predict.

For the downstream analysis, quality check and trimming were needed to
ensure good quality. After using \texttt{fastplong}, it was confirmed
that our flowcell indeed had problems with the digester samples and did
not provide usable reads.

\hypertarget{the-metagenomics-pipeline}{%
\subsubsection{The Metagenomics
Pipeline}\label{the-metagenomics-pipeline}}

\hypertarget{data-preprocessing-and-quality-control}{%
\paragraph{Data Preprocessing and Quality
Control}\label{data-preprocessing-and-quality-control}}

Raw sequencing data was received in FASTQ format and preprocessed using
\texttt{fastplong}. This specialized version of the fastp tool is
optimized for Nanopore long-read data. Adapters and low quality reads
(Phred score \textless{} 20) were trimmed, and reads shorter than 50 bp
were discarded to ensure high-quality data and maintain integrity for
downstream analysis. After the initial trim, data quality was reassessed
using \texttt{fastplong}. It was concluded that no further trimming was
needed, allowing the next step to proceed.

\hypertarget{identifying-microorganisms-using-taxonomic-classification}{%
\paragraph{Identifying Microorganisms Using Taxonomic
Classification}\label{identifying-microorganisms-using-taxonomic-classification}}

The processed FASTQ files were analyzed using \texttt{Kraken2} for
taxonomic classification of microorganisms. \texttt{Kraken2} achieves
high accuracy by matching k-mer sequences (nucleotide sequences). The
output file contains important information including taxonomy ID,
classification status, and the lowest common ancestor list. While
taxonomy IDs reveal microorganism identities, manually searching each ID
would be inefficient. This challenge is addressed in the visualization
step.

\hypertarget{visualization-of-kraken2-results}{%
\paragraph{Visualization of Kraken2
Results}\label{visualization-of-kraken2-results}}

To visualize the taxonomic classification results from \texttt{Kraken2},
we used multiple tools:

\begin{itemize}
\item
  \texttt{KronaTools} created Krona plots showing all taxonomic levels
  from superkingdom to family level, with associated abundances based on
  identified spectra. These plots provide an intuitive representation of
  the findings.
\item
  \texttt{Pavian} provided an interactive browser application for
  analyzing and visualizing metagenomics classification results,
  including:

  \begin{itemize}
  \tightlist
  \item
    Taxonomic classification charts
  \item
    Sankey diagrams
  \item
    An alignment viewer to validate genome matches
  \end{itemize}
\end{itemize}

\hypertarget{other-visualizations}{%
\paragraph{Other Visualizations}\label{other-visualizations}}

\texttt{HUMAnN\ 3.0} profiled the abundance of microbial metabolic
pathways and molecular functions, revealing the metabolic potential of
the lagoon's microbial community. This helped answer the question:
``What are the microorganisms in the lagoon doing or capable of doing?''

\texttt{QIIME2} calculated the Shannon index for alpha diversity,
showing species diversity and distribution in each sample. These
distributions enabled us to build microbiome structures for each lagoon
stage, though specific research questions about the lagoon remain to be
clarified. \# Results

{[}Your results content here{]}

\hypertarget{discussion}{%
\section{Discussion}\label{discussion}}

{[}Your discussion content here{]}

\hypertarget{conclusion}{%
\section{Conclusion}\label{conclusion}}

{[}Your conclusion content here{]}

\hypertarget{funding}{%
\section{Funding}\label{funding}}

{[}Details of funding sources{]}

\hypertarget{acknowledgements}{%
\section{Acknowledgements}\label{acknowledgements}}

These should be included at the end of the text and not in footnotes.
Please ensure you acknowledge all sources of funding.

\hypertarget{refs}{}
\begin{CSLReferences}{1}{0}
\leavevmode\vadjust pre{\hypertarget{ref-Beghini2021}{}}%
Beghini, Francesco, Lauren J. McIver, Aitor Blanco-Mìguez, Leonard
Dubois, Francesco Asnicar, Sagun Maharjan, Ana Mailyan, et al. 2021.
{``Integrating Taxonomic, Functional, and Strain-Level Profiling of
Diverse Microbial Communities with Bio{B}akery 3.''} \emph{eLife} 10:
e65088. \url{https://doi.org/10.7554/eLife.65088}.

\leavevmode\vadjust pre{\hypertarget{ref-Breitwieser2020}{}}%
Breitwieser, Florian P., and Steven L. Salzberg. 2020. {``Pavian:
Interactive Analysis of Metagenomics Data for Microbiome Studies and
Pathogen Identification.''} \emph{Bioinformatics} 36 (4): 1303--4.
\url{https://doi.org/10.1093/bioinformatics/btz715}.

\leavevmode\vadjust pre{\hypertarget{ref-Chen2018}{}}%
Chen, S., Y. Zhou, Y. Chen, and J. Gu. 2018. {``Fastp: An Ultra-Fast
All-in-One {FASTQ} Preprocessor.''} \emph{Bioinformatics} 34 (17):
i884--90. \url{https://doi.org/10.1093/bioinformatics/bty560}.

\leavevmode\vadjust pre{\hypertarget{ref-Ondov2011}{}}%
Ondov, Brian D., Nicholas H. Bergman, and Adam M. Phillippy. 2011.
{``Interactive Metagenomic Visualization in a {Web} Browser.''}
\emph{BMC Bioinformatics} 12 (1): 385.
\url{https://doi.org/10.1186/1471-2105-12-385}.

\leavevmode\vadjust pre{\hypertarget{ref-Wood2019}{}}%
Wood, D. E., J. Lu, and B. Langmead. 2019. {``Improved Metagenomic
Analysis with {Kraken} 2.''} \emph{Genome Biology} 20: 257.
\url{https://doi.org/10.1186/s13059-019-1891-0}.

\end{CSLReferences}

\end{document}
